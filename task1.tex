\documentclass[12pt]{article}
\usepackage[english]{babel}
\usepackage{natbib}
\usepackage{url}
\usepackage[utf8x]{inputenc}
\usepackage{amsmath}
\usepackage{graphicx}
\graphicspath{{images/}}
\usepackage{parskip}
\usepackage{fancyhdr}
\usepackage{vmargin}


\usepackage[normalem]{ulem}
\setmarginsrb{3 cm}{2.5 cm}{3 cm}{2.5 cm}{1 cm}{1.5 cm}{1 cm}{1.5 cm}

\title{Task 1}								% Title
\author{Group 5}								% Author
\date{\today}											% Date

\makeatletter
\let\thetitle\@title
\let\theauthor\@author
\let\thedate\@date
\makeatother

\pagestyle{fancy}
\fancyhf{}
\rhead{\theauthor}
\lhead{\thetitle}
\cfoot{\thepage}

\begin{document}

%%%%%%%%%%%%%%%%%%%%%%%%%%%%%%%%%%%%%%%%%%%%%%%%%%%%%%%%%%%%%%%%%%%%%%%%%%%%%%%%%%%%%%%%%

\begin{titlepage}
	\centering
    \vspace*{0.5 cm}
    \includegraphics[scale = 0.75]{UCT.jpg}\\[1.0 cm]	% University Logo
    \textsc{\LARGE University of Cape Town}\\[1.0 cm]	% University Name
	\textsc{\Large EEE3088F}\\[0.5 cm]				% Course Code
	\textsc{\large Group 5}\\[0.5 cm]				% Course Name
	\rule{\linewidth}{0.2 mm} \\[0.4 cm]
	{ \huge \bfseries \thetitle}\\
	\rule{\linewidth}{0.2 mm} \\[0.5 cm]
	
	\begin{minipage}{0.4\textwidth}
		\begin{flushleft} \large
			\emph{Authors:}\\
      Torsten Babl (leader)\\
      Ebrahim Allie\\
      Yong Hao Chen\\
      Yashiv Fakir\\
      Adam Gild\\
      Jason Hillebrand\\
      Lindokuhle Lubisi\\
      Iviwe Malotana\\
      Samuel Mbiya
      
			\end{flushleft}
			\end{minipage}~
			\begin{minipage}{0.4\textwidth}
			\begin{flushright} \large
			\emph{Student Number:} \\
      BBLTOR001\\
      ALLEBR004\\
      CHNYON001\\
      FKRYAS002\\
      GLDADA002\\
      HLLJAS007\\
      LBSLIN008\\
      MLTIVI001\\
      MBYSAM003\\									% Your Student Number
		\end{flushright}
	\end{minipage}\\[0.5 cm]
	
	{\large \thedate}\\[0.5 cm]
 
	\vfill
	
\end{titlepage}

%%%%%%%%%%%%%%%%%%%%%%%%%%%%%%%%%%%%%%%%%%%%%%%%%%%%%%%%%%%%%%%%%%%%%%%%%%%%%%%%%%%%%%%%%

\tableofcontents
\pagebreak

%%%%%%%%%%%%%%%%%%%%%%%%%%%%%%%%%%%%%%%%%%%%%%%%%%%%%%%%%%%%%%%%%%%%%%%%%%%%%%%%%%%%%%%%%

\section{End User Specifications}

\newpage
\section{Sensing Subsystem} 

% Please add the following required packages to your document preamble:
% \usepackage[normalem]{ulem}
% \useunder{\uline}{\ul}{}
  \begin{center}
    

  \begin{table}[htp]
    \hskip-2.2cm\begin{tabular}{|llll|}
  \hline
  {\underline{ \textbf{SENSING SYSTEM}}} &                                                                            &                      &                                                                                                                                                                                                                                                                                                                                                                                                \\ \hline
  \textbf{Device used:}         & \textbf{\begin{tabular}[c]{@{}l@{}}Communication \\ Standard\end{tabular}} & \textbf{Sample Rate} & \textbf{Reason for sampling}                                                                                                                                                                                                                                                                                                                                                                   \\
  \textbf{GY-91 10DOF:}         &                                                                            &                      &                                                                                                                                                                                                                                                                                                                                                                                                \\
  Magnetometer                  & I2C                                                                        & 100Hz                & \begin{tabular}[c]{@{}l@{}}To measure the change in magnetic \\ field around the current carrying lines.\\  This is to indicate if one or more of \\ the lines has been displaced.\end{tabular}                                                                                                                                                                                                \\
  Accelerometer                 & I2C                                                                        & 1000Hz               & \begin{tabular}[c]{@{}l@{}}Measure the frequency response of \\ the pole as it moves through space \\ due to wind excitation. This is to \\ indicate the physical wear and changes \\ to the pole as a possible warning for\\  a future breakdown.\\ \\ A large spike in acceleration will \\ indicate the pole's movement and \\ collisions with objects \\ (i.e motor vehicle).\end{tabular} \\
  Gyroscope                     & I2C                                                                        & 100Hz                & \begin{tabular}[c]{@{}l@{}}Measure if the pole has changed \\ orientation.\end{tabular}                                                                                                                                                                                                                                                                                                        \\
  Barometer                     & I2C                                                                        & 60Hz                 & \begin{tabular}[c]{@{}l@{}}Measure temperature and elevation \\ above sea level. A drastic increase in \\ temperature warns of fire threats to\\  the pole, altitude measurement can also\\ be used to confirm an incident on the pole.\end{tabular}                                                                                                                                           \\ \hline
  \end{tabular}
  \end{table}
\end{center}
\newpage  
\section{Micro-controller Subsystem}

\newpage
\section{Power Supply Subsystem}

\begin{center}
\begin{table}[tbh]
\begin{tabular}{|l|l|l|}
\hline
\multicolumn{1}{|c|}{\textbf{\begin{tabular}[c]{@{}c@{}}Design\\   Parameters\end{tabular}}} &
  \multicolumn{1}{c|}{\textbf{Rating}} &
  \multicolumn{1}{c|}{\textbf{\begin{tabular}[c]{@{}c@{}}Components and\\   Interface Requirement\end{tabular}}} \\ \hline
\begin{tabular}[c]{@{}l@{}}Power\\   under max load\end{tabular} &
  \begin{tabular}[c]{@{}l@{}}1 W\\   under load\end{tabular} &
  n/a \\ \hline
Stable supply voltage &
  5 V input supply &
  Power source \\ \hline
Step down the supply voltage &
  3.3 V step down supply &
  DC-DC converter \\ \hline
DC supply voltages &
  5 V and 3.3 V DC &
  \begin{tabular}[c]{@{}l@{}}Fed into PCB (2 supply lines\\   on PCB)\end{tabular} \\ \hline
Long lasting power source &
  5 V rating &
  n/a \\ \hline
\begin{tabular}[c]{@{}l@{}}Some way to recharge the power\\   source\end{tabular} &
  Possibly AC to DC &
  \begin{tabular}[c]{@{}l@{}}Solar or Directly from pole\\   line\end{tabular} \\ \hline
\begin{tabular}[c]{@{}l@{}}Choose affordable but\\   efficient components\end{tabular} &
  n/a &
  \begin{tabular}[c]{@{}l@{}}internal PCB and external\\   source\end{tabular} \\ \hline
\begin{tabular}[c]{@{}l@{}}A rechargeable power storage\\   source\end{tabular} &
  minimum 500 cycles &
  Power source \\ \hline
\begin{tabular}[c]{@{}l@{}}Components should not be\\   over-sized\end{tabular} &
  n/a &
  Fit into packaging \\ \hline
\begin{tabular}[c]{@{}l@{}}Regulate from AC to DC\\   effectively\end{tabular} &
  19 kV/11 kV to 5 V &
  \begin{tabular}[c]{@{}l@{}}From Pole line to the power\\   source\end{tabular} \\ \hline
\end{tabular}
\end{table}
\end{center}


\newpage
\section{Acceptance Test Procedures}

\newpage
\section{Minutes of Meetings}

\newpage
%\bibliographystyle{plain}
%\bibliography{biblist}

\end{document}