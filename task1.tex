\documentclass[12pt]{article}
\usepackage[english]{babel}
\usepackage{natbib}
\usepackage{url}
\usepackage[utf8x]{inputenc}
\usepackage{amsmath}
\usepackage{graphicx}
\graphicspath{{images/}}
\usepackage{parskip}
\usepackage{fancyhdr}
\usepackage{vmargin}


\usepackage[normalem]{ulem}
\setmarginsrb{3 cm}{2.5 cm}{3 cm}{2.5 cm}{1 cm}{1.5 cm}{1 cm}{1.5 cm}

\title{Task 1}								% Title
\author{Group 5}								% Author
\date{\today}											% Date

\makeatletter
\let\thetitle\@title
\let\theauthor\@author
\let\thedate\@date
\makeatother

\pagestyle{fancy}
\fancyhf{}
\rhead{\theauthor}
\lhead{\thetitle}
\cfoot{\thepage}

\begin{document}

%%%%%%%%%%%%%%%%%%%%%%%%%%%%%%%%%%%%%%%%%%%%%%%%%%%%%%%%%%%%%%%%%%%%%%%%%%%%%%%%%%%%%%%%%

\begin{titlepage}
	\centering
    \vspace*{0.5 cm}
    \includegraphics[scale = 0.75]{UCT.jpg}\\[1.0 cm]	% University Logo
    \textsc{\LARGE University of Cape Town}\\[1.0 cm]	% University Name
	\textsc{\Large EEE3088F}\\[0.5 cm]				% Course Code
	\textsc{\large Group 5}\\[0.5 cm]				% Course Name
	\rule{\linewidth}{0.2 mm} \\[0.4 cm]
	{ \huge \bfseries \thetitle}\\
	\rule{\linewidth}{0.2 mm} \\[0.5 cm]
	
	\begin{minipage}{0.4\textwidth}
		\begin{flushleft} \large
			\emph{Authors:}\\
      Torsten Babl (leader)\\
      Ebrahim Allie\\
      Yong Hao Chen\\
      Yashiv Fakir\\
      Adam Gild\\
      Jason Hillebrand\\
      Lindokuhle Lubisi\\
      Iviwe Malotana\\
      Samuel Mbiya
      
			\end{flushleft}
			\end{minipage}~
			\begin{minipage}{0.4\textwidth}
			\begin{flushright} \large
			\emph{Student Number:} \\
      BBLTOR001\\
      ALLEBR004\\
      CHNYON001\\
      FKRYAS002\\
      GLDADA002\\
      HLLJAS007\\
      LBSLIN008\\
      MLTIVI001\\
      MBYSAM003\\									% Your Student Number
		\end{flushright}
	\end{minipage}\\[0.5 cm]
	
	{\large \thedate}\\[0.5 cm]
 
	\vfill
	
\end{titlepage}

%%%%%%%%%%%%%%%%%%%%%%%%%%%%%%%%%%%%%%%%%%%%%%%%%%%%%%%%%%%%%%%%%%%%%%%%%%%%%%%%%%%%%%%%%

\tableofcontents
\pagebreak

%%%%%%%%%%%%%%%%%%%%%%%%%%%%%%%%%%%%%%%%%%%%%%%%%%%%%%%%%%%%%%%%%%%%%%%%%%%%%%%%%%%%%%%%%

\section{Definitions}

Node - A single sensing device on a pole that meshes with other nodes to form the sensing network

\section{End User Specifications}

The aim of the comissioned sensing and reporting system is to identify and report faults on 33kV transmission lines 
that do not trigger an earth leakage or overcurrent fault condition on existing switching systems, yet still pose 
a safety risk to the public. Such faults are often caused by storm damage, motor/construction vehicle impact and malicious
damage to public property generally caused by theft. In addition to post-fault reporting the early recognition of 
structural degridation can potentially save the Utility money by allowing planned and targeted maintainence to prevent 
future unexpected faults.

\subsection{Definition of desired detactable fault conditions}
The following fault conditoins are desirted to be detected by the comissioned system:

\begin{table}[!ht]
  \caption{Definition of desired detactable fault conditions}
  \begin{center}
  \begin{tabular}{|l|l|l|}
      \hline
      \textbf{Image} & \textbf{Name} & \textbf{Description} \\
      \hline
      One & Phase failure & Loss of energy of one or more lines due to faults between the node and the supply \\
       & Line detachment & One or more lines are detachted from the isolator on the A-frame while the pole remains
       upright \\
       & Pole dislodged and/or rotated & Pole is no longer upright or loses orientated through forces on the lines or pole itself \\
       & Pole failure & The structure of the pole is compromised due to material failure \\
       & Pole fracture & The pole remains upright, the strength is however compromised due to any number of fractures \\
       & Pole and node failure due to damaging force & A damaging force is strong enough to damage both the pole and the
       node\\
       & Fire in vicinity of transmission line & \\
      \hline

  \end{tabular}      
\end{center}
\label{tab:summary_measurments}
\end{table}

\newpage
\section{Sensing Subsystem} 

% Please add the following required packages to your document preamble:
% \usepackage[normalem]{ulem}
% \useunder{\uline}{\ul}{}
  \begin{center}
    

  \begin{table}[htp]
    \hskip-2.2cm\begin{tabular}{|llll|}
  \hline
  {\underline{ \textbf{SENSING SYSTEM}}} &                                                                            &                      &                                                                                                                                                                                                                                                                                                                                                                                                \\ \hline
  \textbf{Device used:}         & \textbf{\begin{tabular}[c]{@{}l@{}}Communication \\ Standard\end{tabular}} & \textbf{Sample Rate} & \textbf{Reason for sampling}                                                                                                                                                                                                                                                                                                                                                                   \\
  \textbf{}         &                                                                            &                      &                                                                                                                                                                                                                                                                                                                                                                                                \\
  Magnetometer                  & I2C                                                                        & 100Hz                & \begin{tabular}[c]{@{}l@{}}To measure the change in magnetic \\ field around the current carrying lines.\\  This is to indicate if one or more of \\ the lines has been displaced.\end{tabular}                                                                                                                                                                                                \\
  Accelerometer                 & I2C                                                                        & 1000Hz               & \begin{tabular}[c]{@{}l@{}}Measure the frequency response of \\ the pole as it moves through space \\ due to wind excitation. This is to \\ indicate the physical wear and changes \\ to the pole as a possible warning for\\  a future breakdown.\\ \\ A large spike in acceleration will \\ indicate the pole's movement and \\ collisions with objects \\ (i.e motor vehicle).\end{tabular} \\
  Gyroscope                     & I2C                                                                        & 100Hz                & \begin{tabular}[c]{@{}l@{}}Measure if the pole has changed \\ orientation.\end{tabular}                                                                                                                                                                                                                                                                                                        \\
  Barometer                     & I2C                                                                        & 60Hz                 & \begin{tabular}[c]{@{}l@{}}Measure temperature and elevation \\ above sea level. A drastic increase in \\ temperature warns of fire threats to\\  the pole, altitude measurement can also\\ be used to confirm an incident on the pole.\end{tabular}                                                                                                                                           \\ \hline
  \end{tabular}
  \end{table}
\end{center}
\newpage  
\section{Micro-controller Subsystem}

\newpage
\section{Power Supply Subsystem}

\newpage
\section{Acceptance Test Procedures}

\newpage
\section{Minutes of Meetings}

\newpage
%\bibliographystyle{plain}
%\bibliography{biblist}

\end{document}