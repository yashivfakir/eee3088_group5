\documentclass[12pt]{article}
\usepackage[english]{babel}
\usepackage{natbib}
\usepackage{url}
\usepackage[utf8x]{inputenc}
\usepackage{amsmath}
\usepackage{graphicx}
\graphicspath{{images/}}
\usepackage{parskip}
\usepackage{fancyhdr}
\usepackage{vmargin}
\usepackage{longtable}
\usepackage{multirow}

\usepackage{opensans} 
\renewcommand{\familydefault}{\sfdefault}

\usepackage[normalem]{ulem}
\setmarginsrb{3 cm}{2.5 cm}{3 cm}{2.5 cm}{1 cm}{1.5 cm}{1 cm}{1.5 cm}

\title{Task 1}								% Title
\author{Group 5}								% Author
\date{\today}											% Date

\makeatletter
\let\thetitle\@title
\let\theauthor\@author
\let\thedate\@date
\makeatother

\pagestyle{fancy}
\fancyhf{}
\rhead{\theauthor}
\lhead{\thetitle}
\cfoot{\thepage}

\begin{document}


%%%%%%%%%%%%%%%%%%%%%%%%%%%%%%%%%%%%%%%%%%%%%%%%%%%%%%%%%%%%%%%%%%%%%%%%%%%%%%%%%%%%%%%%%

\begin{titlepage}
	\centering
    \vspace*{0.5 cm}
    \includegraphics[scale = 0.75]{UCT.jpg}\\[1.0 cm]	% University Logo
    \LARGE University of Cape Town\\[1.0 cm]	% University Name
	\Large EEE3088F\\				% Course Code
	\large Group 5\\				% Course Name
	\rule{\linewidth}{0.2 mm} \\[0.4 cm]
	{ \huge \bfseries \thetitle}\\
	\rule{\linewidth}{0.2 mm} \\[0.5 cm]
	
	\begin{minipage}{0.4\textwidth}
		\begin{flushleft} \large
			\emph{Authors:}\\
      Torsten Babl (leader)\\
      Ebrahim Allie\\
      Yong Hao Chen\\
      Yashiv Fakir\\
      Adam Gild\\
      Jason Hillebrand\\
      Lindokuhle Lubisi\\
      Iviwe Malotana\\
      Samuel Mbiya
      
			\end{flushleft}
			\end{minipage}~
			\begin{minipage}{0.4\textwidth}
			\begin{flushright} \large
			\emph{Student Number:} \\
      BBLTOR001\\
      ALLEBR004\\
      CHNYON001\\
      FKRYAS002\\
      GLDADA002\\
      HLLJAS007\\
      LBSLIN008\\
      MLTIVI001\\
      MBYSAM003\\									% Your Student Number
		\end{flushright}
	\end{minipage}\\[0.5 cm]
	
	{\large \thedate}\\[0.5 cm]
 
	\vfill
	
\end{titlepage}

%%%%%%%%%%%%%%%%%%%%%%%%%%%%%%%%%%%%%%%%%%%%%%%%%%%%%%%%%%%%%%%%%%%%%%%%%%%%%%%%%%%%%%%%%

\tableofcontents
\pagebreak

%%%%%%%%%%%%%%%%%%%%%%%%%%%%%%%%%%%%%%%%%%%%%%%%%%%%%%%%%%%%%%%%%%%%%%%%%%%%%%%%%%%%%%%%%

\section{Definitions}

Node - A single sensing device on a pole that meshes with other nodes to form the sensing network

\section{End User Specifications}

The aim of the comissioned sensing and reporting system is to identify and report faults on 33kV transmission lines 
that do not trigger an earth leakage or overcurrent fault condition on existing switching systems, yet still pose 
a safety risk to the public. Such faults are often caused by storm damage, motor/construction vehicle impact and 
malicious damage to public property generally caused by theft. In addition to post-fault reporting the early recognition
of structural degridation can potentially save the Utility money by allowing planned and targeted maintainence to prevent 
future unexpected faults.

\subsection{Definition of desired detactable fault conditions}
The following fault conditoins are desirted to be detected by the comissioned system:

\begin{center}
\begin{table}[htp]
  %\caption{Definition of desired detactable fault conditions}
  
  \hskip-2.2cm\begin{tabular}{|p{4cm}|p{4cm}|p{10cm}|}
      \hline
      \textbf{Image} & \textbf{Name} & \textbf{Description} \\
      \hline
      One & Phase failure & Loss of energy of one or more lines due to faults between the node and the supply \\
       & Line detachment & One or more lines are detachted from the isolator on the A-frame while the pole remains
       upright \\
       & Pole dislodged and/or rotated & Pole is no longer upright or loses orientated through forces on the lines or pole itself \\
       & Pole failure & The structure of the pole is compromised due to material failure \\
       & Pole fracture & The pole remains upright, the strength is however compromised due to any number of fractures \\
       & Pole and node failure due to damaging force & A damaging force is strong enough to damage both the pole and the
       node\\
       & Fire in vicinity of transmission line & \\
      \hline

  \end{tabular}    

\label{tab:summary_measurments}
\end{table}
\end{center}

% TODO: Put these into tables later on

\subsection{Reporting requirements for faults}
Faults that compromise the safety of the public should be reported as fast as possible and in no more than 10 minutes.

Fault conditions that  precede an immminent future failure should be reported within 24 hours of the condition being
detected.

\subsection{Durability and cost of operation}
The sensing devices (nodes) should not be required to be serviced for 10 years after installation.

The nodes should not rely on subsription services (such as GSM) for operation.

\subsection{Component cost}
A bill of materials of single device should exceed 250 ZAR.

\newpage
\section{Data interface requirements}
\subsection{Sensing Subsystem} 

% Please add the following required packages to your document preamble:
% \usepackage[normalem]{ulem}
% \useunder{\uline}{\ul}{}
  \begin{center}
    

  \begin{table}[htp]
    \hskip-2.2cm\begin{tabular}{|llll|}
  \hline
  \textbf{Device used:}         & \textbf{\begin{tabular}[c]{@{}l@{}}Communication \\ Standard\end{tabular}} & \textbf{Sample Rate} & \textbf{Reason for sampling}                                                                                                                                                                                                                                                                                                                                                                   \\
  \hline
    \textbf{}         &                                                                            &                      &                                                                                                                                                                                                                                                                                                                                                                                                \\
  Magnetometer                  & I2C                                                                        & 100Hz                & \begin{tabular}[c]{@{}l@{}}To measure the change in magnetic \\ field around the current carrying lines.\\  This is to indicate if one or more of \\ the lines has been displaced.\end{tabular}                                                                                                                                                                                                \\
  Accelerometer                 & I2C                                                                        & 1000Hz               & \begin{tabular}[c]{@{}l@{}}Measure the frequency response of \\ the pole as it moves through space \\ due to wind excitation. This is to \\ indicate the physical wear and changes \\ to the pole as a possible warning for\\  a future breakdown.\\ \\ A large spike in acceleration will \\ indicate the pole's movement and \\ collisions with objects \\ (i.e motor vehicle).\end{tabular} \\
  Gyroscope                     & I2C                                                                        & 100Hz                & \begin{tabular}[c]{@{}l@{}}Measure if the pole has changed \\ orientation.\end{tabular}                                                                                                                                                                                                                                                                                                        \\
  Barometer                     & I2C                                                                        & 60Hz                 & \begin{tabular}[c]{@{}l@{}}Measure temperature and elevation \\ above sea level. A drastic increase in \\ temperature warns of fire threats to\\  the pole, altitude measurement can also\\ be used to confirm an incident on the pole.\end{tabular}                                                                                                                                           \\ \hline
  \end{tabular}
  \end{table}
\end{center} 
\subsection{Micro-controller Subsystem}
Due to the requirements for memory in the system, a 32-bit microprocessor will be used for data capture, processing and information transmission. A compromise to 16/8 bit processor will not be necessary as lower priced 32-bit microprocessors are priced at around R12, 
a relatively small fraction of the total budget per unit.\\
The chosen uC: \underline{STM32F373C8}\\

\begin{longtable}[c]{|l|l|l|}
  \hline
  \textbf{Processor}          & \multicolumn{2}{l|}{\textbf{Specifications}}                                                                                                                                  \\ \hline
  \endfirsthead
  %
  \endhead
  %
  \multirow{13}{*}{STM32F373C8} & Core:                        & ARM®32-bit Cortex®-M4 CPU                                                                                                             \\ \cline{2-3}
                     & \textit{Frequency:}                   & Up to 72Mhz                                                                                                                           \\ \cline{2-3} 
                     & \textit{Storage:}                     & \begin{tabular}[c]{@{}l@{}}64 to 256 Kbytes of Flash memory \\ (128Kbytes chosen for this device)\end{tabular}                        \\ \cline{2-3} 
                     & \textit{Voltage range:}               & 2.0 to 3.6 V                                                                                                                          \\ \cline{2-3} 
                     & \textit{Voltage detector:}            & Programmable voltage detector (PVD)                                                                                                   \\ \cline{2-3} 
                     & \textit{Low power modes:}             & Sleep, Stop, Standby                                                                                                                  \\ \cline{2-3} 
                     & \textit{Clock:}                       & 4 to 32 MHz crystal oscillator                                                                                                        \\ \cline{2-3} 
                     &\textit{ADCs:}                        & Three 16-bit Sigma Delta ADCs                                                                                                         \\ \cline{2-3} 
                     & \multirow{5}{*}{\textit{Interfaces:}} & \begin{tabular}[c]{@{}l@{}}Two I2Cs supporting Fast Mode Plus\\  (1 Mbit/s) with 20 mA current sink.\end{tabular} \\ \cline{3-3} 
                     &                              & Three USARTs supporting synchronous mode                                                                                              \\ \cline{3-3} 
                     &                              & \begin{tabular}[c]{@{}l@{}}Three SPIs (18 Mbit/s) with 4 to 16\\  programmable bit frames\end{tabular}                                \\ \cline{3-3} 
                     &                              & HDMI-CEC bus interface                                                                                                                \\ \cline{3-3} 
                     &                              & USB 2.0 full speed interface                                                                                                          \\ \hline
  \end{longtable}


\subsection{Wireless Subsystem}

\begin{center}
  \begin{table}[!htb]
    %\caption{Definition of desired detactable fault conditions}
    
    \hskip-2.2cm\begin{tabular}{|p{8cm}|p{10cm}|}
        \hline
        \textbf{Specification} & \textbf{Description} \\
        \hline
        Communication to main micro-controller & Asyncronous serial communication. Likely USART 3.3V logic levels at low
        baud rates. (4800 or 9600 baud) \\[0.3cm]
        
        RF hardware components & IoT transciever module simialr to the Ubiik Weightless End Device Module \\[0.3cm]
        RF band used & 868.0 - 868.6 MHz band allocated by ICASA for Non-specific Short-Range-Devices (SRDs) limited to 
        25mW (14dBm) and duty cycles $<$1\%. \\[0.3cm]
        Antenna system & 868 MHz dipole antenna printed onto the main PCB1 \\[0.3cm]
        Protocol & The open IoT network standard Weightless-P will be used under the Ubiik hardware
        implementation \\[0.3cm]
        Network topology & A peer to peer (P2P) network topology will be implemented with nodes being able to communicate
        with a minimum of three nodes backwards and forwards for redundancy.\\[0.3cm]

        \hline

    \end{tabular}    
    
    \label{tab:summary_measurments}
   \end{table}
\end{center}


\subsection{Power Supply Subsystem}

\begin{center}
  \begin{table}[!htb]
    %\caption{Definition of desired detactable fault conditions}
    
    \hskip-2.2cm\begin{tabular}{|p{10cm}|p{4cm}|p{4cm}|}
        \hline
        \textbf{Specification} & \textbf{Data} & \textbf{Interface} \\
        \hline
        Must be able to supply the needed power to all the sub-systems & Rated Power & Analogue outputs to uC for supply voltage \\[0.3cm]
        Disconnect power to unused sub-systems & Pre-programmed microcontroller instructions & Wireless and uC sub-systems\\[0.3cm]
        At least one backup supply is needed & Rated Power & Power supply rails\\[0.3cm]
        Should be able to withstand extreme conditions i.e. fire, thunderstorms, heat waves etc. & None & Exterior hardware\\[0.3cm]
        Should not be too big. The weight of the component is important & Specified final dimensions & None\\[0.3cm]
        Supply the correct voltages to the microcontroller and sensors & Rated voltages & None\\[0.3cm]
        Inexpensive materials to be used & Total BOM budget & None\\[0.3cm]
        Simple materials are to be used to ease repairs and maintenance & None & None \\[0.3cm]

        \hline

    \end{tabular}    
    
    \label{tab:summary_measurments}
   \end{table}
\end{center}

\newpage
\section{Acceptance Test Procedures}
\subsection{Sensing}
\begin{center}
  \begin{table}[!htb]
    %\caption{Definition of desired detactable fault conditions}
    
    \hskip-2.2cm\begin{tabular}{|p{8cm}|p{10cm}|}
        \hline
        \textbf{Test} & \textbf{Pass criteria} \\
        \hline

        Phase failure & The magnetometer reading reliably indicates the loss of power in the phase\\[0.3cm]
        Line detached & The magnetometer reading reliably indicates the loss of power in the phase\\[0.3cm]
        Pole dislodged & The accelerometer and gyroscope indicate a change in pole orientation\\[0.3cm]
        Pole failure, lying flat & The accelerometer and gyroscope indicate that the pole has fallen\\[0.3cm]
        Pole failure, top section upright & The accelerometer indicates that the pole has fallen\\[0.3cm]
        Pole fracture & The accelerometer indicates a significant change in the resonant frequency of the pole \\[0.3cm]

        \hline

      \end{tabular}    
      
      \label{tab:summary_measurments}
     \end{table}
  \end{center}

\subsection{Wireless}
\begin{center}
  \begin{table}[!htb]
    %\caption{Definition of desired detactable fault conditions}
    
    \hskip-2.2cm\begin{tabular}{|p{8cm}|p{10cm}|}
        \hline
        \textbf{Test} & \textbf{Pass criteria} \\
        \hline

        Reporting latency & A fault from various points in the transmission line takes no longer than 10 minutes to be
        reported \\[0.3cm]
        Reporting overload & Any number of concurrent faults does not hinder the highest priority fault to be reported to
        the distribution station \\[0.3cm]
        Node failure redundancy & Two consecutive node failures to not break the line of communication to the
        distribution station \\[0.3cm]


        \hline

      \end{tabular}    
      
      \label{tab:summary_measurments}
     \end{table}
  \end{center}

\subsection{Micro-controller}

-\underline{Processing capabilities test:} The uC needs to be able to process 5 sets of input data (main voltage source, backup voltage source, accelerometer, magnetometer, temperature sensor) within 2 seconds consistently for long periods of time (potentially weeks/months) without overheating or degrading(too much). Run a stress test on the uC with the maximum load capable from the system.\\
\\
-\underline{Fail case operation testing:} The uC needs to be be able to continue functioning even with a break or deformation in the pole, ie placement and packaging needs to ensure that the uC sub-system can withstand the forces applied on it during this case of failure.Test the device against impulse forces with increasing strength. This would allow for comparison with the required strength of the chip.\\
\\
-\underline{Synchronization test:} The ADC's transmission rate of data needs to stay in sync with the digital readings from the accelerometer and magnetometer such that the fault detection packages are used efficiently on all 5 data sets, rather than 1 by 1. Test by running ranges of possible data through all 5 data channels, including analog signals at similar frequencies to the sensors and voltage sources, and testing that processing is done synchronously.\\
\\
- \underline{Storage test:} The data capture and storage of the uC sub-system need to be tested.Input data needs to be recorded accurately within certain margins of accuracy. Test by feeding predefined signals into the sub-system and reading the values stored in storage (flash storage) ,comparing for the accuracy expected from the system.\\ 
\\
-\underline{Environmental temperature test:} During operation the uC could experience either high or low temperatures , as well as sharp temperature fluctuations depending on the location of the device. In order to test this case , the device should be tested against three cases (heating,cooling and sharp temperature gradients).\\
\\
-\underline{Rain test:} Water and wind might be introduced to the uC system in rainy/stormy weather if the package is compromised.The PCB should be protected with a hydrophobic coating. The ability for the device to stay operational during these periods would then be assured. Use laboratory UV light, wind and rain sources to test resistance against stormy conditions.\\

\subsection{Power supply}
\begin{center}
  \begin{table}[!htb]
    %\caption{Definition of desired detactable fault conditions}
    
    \hskip-2.2cm\begin{tabular}{|p{8cm}|p{10cm}|}
        \hline
        \textbf{Test} & \textbf{Pass criteria} \\
        \hline

        One & Two \\

        \hline

      \end{tabular}    
      
      \label{tab:summary_measurments}
     \end{table}
  \end{center}

\subsection{Complete system}
\begin{center}
  \begin{table}[!htb]
    %\caption{Definition of desired detactable fault conditions}
    
    \hskip-2.2cm\begin{tabular}{|p{8cm}|p{10cm}|}
        \hline
        \textbf{Test} & \textbf{Pass criteria} \\
        \hline

        One & Two \\

        \hline

      \end{tabular}    
      
      \label{tab:summary_measurments}
     \end{table}
  \end{center}

\newpage
\section{Minutes of Meetings}
\subsection{Meeting 1: 14 February 2020}
14 February 2020, 12:00, NEB Foyer

\textbf{Present}\\
Torsten Babl, Yashiv Fakir, Jason Hillebrand, Samuel Mbiya, Adam Gild, Yong Hao Chen, Ebrahim Allie

\textbf{Excused}\\
Lindokuhle Lubisi, Iviwe Malotana

\textbf{Topics}
\paragraph[short]{1.} Torsten Babl elected as group leader.
\paragraph[short]{2.} Subsections allocated as follows:

Power: Yashiv, Yong

Micro-controller:	Ebrahim, Adam

Wireless:	Samuel,	Torsten

Sensing: Jason, Iviwe (Assigned as was not present at meeting)
  
Unasigned: Lindokuhle

\paragraph[short]{3.} Communication and work space for project decided upon. Each person is responisble for their own education to be able to use the tools that will be used in the project they are not familiar with. A channel will be set up in Microsoft teams to help anyone struggling.  
Microsoft Teams to be used for all project communication to streamline communication.
Github repsitory to be used to house all project work.
Project to be typed and compiled in LaTeX document in the Git repository.
\newpage

\subsection{Meeting 2: 21 February 2020}
21 February 2020, 13:00, NEB Foyer

\textbf{Present}\\
Torsten Babl, Yashiv Fakir, Jason Hillebrand, Samuel Mbiya, Adam Gild, Iviwe Malotana, Lindokuhle Lubisi

\textbf{Excused}\\
Yong Hao, Ebrahim Allie

\textbf{Topics}
\paragraph[short]{1.} General design decisions taken:\\
Sensing subsystem:
\begin{itemize}
  \item Accelerometer
  \item Magnetometer
  \item Gyroscope
  \item Thermometer
  \item Power supply voltage
\end{itemize}

Micro-controller:
\begin{itemize}
  \item Have low power mode
  \item Communicate using I2C and/or SPI
  \item Low sampling rate required
\end{itemize}

Power supply:
\begin{itemize}
  \item Induce current from the transmission lines to charge a battery
  \item Subsystems to draw power directly from battery
\end{itemize}

\paragraph[short]{2.} Write up:\\
Each sub-team is to complete their respective sections of the write up.
Subsections to be submitted by the 23rd of February at 14h00.

\paragraph[short]{3.} General acceptance tests:\\
All to contribute to acceptance test section. Use a potential fault that the system is to detect and state the
systems response to this.

%\bibliographystyle{plain}
%\bibliography{biblist}

\end{document}