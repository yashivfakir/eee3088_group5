\documentclass[12pt]{article}
\usepackage[english]{babel}
\usepackage{natbib}
\usepackage{url}
\usepackage[utf8x]{inputenc}
\usepackage{amsmath}
\usepackage{graphicx}
\graphicspath{{images/}}
\usepackage{parskip}
\usepackage{fancyhdr}
\usepackage{vmargin}
\usepackage{longtable}
\usepackage{multirow}

\usepackage{opensans} 
\renewcommand{\familydefault}{\sfdefault}

\usepackage[normalem]{ulem}
\setmarginsrb{3 cm}{2.5 cm}{3 cm}{2.5 cm}{1 cm}{1.5 cm}{1 cm}{1.5 cm}

\title{Task 2 Presubmission}								% Title
\author{Group 5}								% Author
\date{\today}											% Date

\makeatletter
\let\thetitle\@title
\let\theauthor\@author
\let\thedate\@date
\makeatother

\pagestyle{fancy}
\fancyhf{}
\rhead{\theauthor}
\lhead{\thetitle}
\cfoot{\thepage}

\begin{document}


%%%%%%%%%%%%%%%%%%%%%%%%%%%%%%%%%%%%%%%%%%%%%%%%%%%%%%%%%%%%%%%%%%%%%%%%%%%%%%%%%%%%%%%%%

\begin{titlepage}
	\centering
    \vspace*{0.5 cm}
    \includegraphics[scale = 0.75]{UCT.jpg}\\[1.0 cm]	% University Logo
    \LARGE University of Cape Town\\[1.0 cm]	% University Name
	\Large EEE3088F\\				% Course Code
	\large Group 5\\				% Course Name
	\rule{\linewidth}{0.2 mm} \\[0.4 cm]
	{ \huge \bfseries \thetitle}\\
	\rule{\linewidth}{0.2 mm} \\[0.5 cm]
	
	\begin{minipage}{0.4\textwidth}
		\begin{flushleft} \large
			\emph{Authors:}\\
      Torsten Babl (leader)\\
      Ebrahim Allie\\
      Yong Hao Chen\\
      Yashiv Fakir\\
      Adam Gild\\
      Jason Hillebrand\\
      Lindokuhle Lubisi\\
      Iviwe Malotana\\
      Samuel Mbiya
      
			\end{flushleft}
			\end{minipage}~
			\begin{minipage}{0.4\textwidth}
			\begin{flushright} \large
			\emph{Student Number:} \\
      BBLTOR001\\
      ALLEBR004\\
      CHNYON001\\
      FKRYAS002\\
      GLDADA002\\
      HLLJAS007\\
      LBSLIN008\\
      MLTIVI001\\
      MBYSAM003\\									% Your Student Number
		\end{flushright}
	\end{minipage}\\[0.5 cm]
	
	{\large \thedate}\\[0.5 cm]
 
	\vfill
	
\end{titlepage}

%%%%%%%%%%%%%%%%%%%%%%%%%%%%%%%%%%%%%%%%%%%%%%%%%%%%%%%%%%%%%%%%%%%%%%%%%%%%%%%%%%%%%%%%%

\tableofcontents
\pagebreak

%%%%%%%%%%%%%%%%%%%%%%%%%%%%%%%%%%%%%%%%%%%%%%%%%%%%%%%%%%%%%%%%%%%%%%%%%%%%%%%%%%%%%%%%%

\section{Executive Summary}

\section{Group-member Contributions}
\newpage
\section{User Requirements and Acceptance Test Procedures}

The aim of the commissioned sensing and reporting system is to identify and report faults on 33kV transmission lines 
that do not trigger an earth leakage or overcurrent fault condition on existing switching systems, yet still pose 
a safety risk to the public. Such faults are often caused by storm damage, motor/construction vehicle impact and 
malicious damage to public property generally caused by theft. In addition to post-fault reporting the early recognition
of structural degradation can potentially save the Utility money by allowing planned and targeted maintenance to prevent 
future unexpected faults.

Below follows the Utility's user requirements and the acceptance tests thereof.

\subsection{User stories}
The commissioned product is envisioned to fulfil the following tasks:\newline
\emph{See the following subsections for detailed scope of key terms.}
\begin{itemize}
  \item The product must report a fault occurring along the transmission line shortly after it occurs.
  \item The product must, upon request, report data useful to assess the structural integrity of the pole.
\end{itemize}

\subsection{User Requirements of The Utility}
The following faults must be detectable:

\begin{center}
  \begin{table}[htp!]
    \caption{Table of Faults to Detect}
    
    \hskip-2.2cm\begin{tabular}{|p{2cm}|p{4cm}|p{8cm}|p{4cm}|}
        \hline
        \textbf{Fault No.} & \textbf{Fault Name} & \textbf{Detailed Description} & \textbf{Illustration} \\
        \hline
        F01 & Phase failure & Loss  of  energy  of  one  or  more  lines  due  to  faults
        between the node and the supply. & - \\\hline

        F02 & Line detachment & One or more lines are detached from the isolator on the A-frame while the pole remains
        upright. & - \\\hline

        F03 & Pole dislodgment and/or rotation & Pole is no longer upright or loses orientation through forces on the
        lines or pole itself. & - \\\hline

        F00 & Strong Impact to pole & An unusually strong impact to the pole is observed. & - \\\hline

        F00 & Sudden movement & The top of the pole is still upright and oriented as before but has moved out of
        position. i.e. Top segment of pole is suspended by the lines. & - \\\hline
        %Don't like these two
        F04 & Possible structural fatigue of pole & The structure of the pole is compromised due to material fatigue. & - \\\hline

        F06 & Pole and node failure due to damaging force & A damaging force is strong enough to damage both the pole
        and the node. & - \\\hline
  
    \end{tabular}    
  
  \label{tab:faults}
  \end{table}
\end{center}

\begin{center}
  \begin{table}[htp!]
    \caption{Table of General User Requirements}
    
    \hskip-2.2cm\begin{tabular}{|p{3cm}|p{2cm}|p{4cm}|p{9cm}|}
        \hline
        \textbf{Requirement Category} & \textbf{Req. No.} & \textbf{Requirement Name} & \textbf{Detailed Description} \\
        \hline
        One & R00 & Reporting time & A detected fault must be reported within 10 minutes of the incident \\\hline
        One & R00 & Product lifespan & 90\% of deployed devices must still be functional after 10 years of installation \\\hline
        %Add BOM to list of definitions
        One & R00 & Device cost & The BOM cost of a single device should not exceed 250 ZAR \\\hline
        One & R00 & Additional infrastructure & If any additional infrastructure is needed it should not add significantly
        to the cost of rollout \\\hline
        %Add ISM band to list of Definitions
        One & R00 & Communication bands & All wireless communications should use ISM bands permitted in South Africa
        to ease certification and decrease cost. \\\hline
  
    \end{tabular}    
  
  \label{tab:faults}
  \end{table}
\end{center}

\subsection{Acceptance Test procedures for User Requirements}
The user requirements set above shall be verified according to the following test criteria.


\section{System Design and Overview}

\section{Sensing Subsystem Design}

\section{Microcontroller Subsystem Design}

\section{Wireless Communications Design}

\section{Power Supply Design}

\section{Additional Design Components}

\section{References}

\section{Appendices}

\end{document}